\documentclass[a4paper]{article}
\usepackage[margin=0cm]{geometry}
\usepackage{tikz}
\usepackage{forloop}


% Name of the module
\newcommand\modulename{Temperature and Matter}

% Disclaimer
\newcommand\disclaimer{The attached work which is submitted for assessment, represents my own effort. I have not copied from anyone else.}

% Uncomment for sans font
% \renewcommand{\familydefault}{\sfdefault}
% \renewcommand\textsc[1]{\uppercase{#1}}

\begin{document}

% Size of the box we are working in
\def\x{21}
\def\y{6}

% limits of the smaller box
\def\xleft{3}
\def\xright{18}


% To make the default font very readable
\large

% a little space beforeo the first slip
\linespread{0}
% For loop is for multiple slips
\newcounter{iterator}
\forloop{iterator}{1}{\value{iterator}<6}{%
  \noindent\begin{tikzpicture}[y=-0.925cm]
    \linespread{0.95}
    \draw [white] (0,0) -- (\x,\y); % the dimensions of this box
    \draw [dashed] (1.5,0) -- (1.5,\y) node [midway,sloped,above] {\footnotesize \textsc{Staple twice to the left of this line}};
    \draw (\x/2,0.5) node  {\large Physics Problem Sheet};
    \draw (\x/2,1.05) node  {\huge \textsc{\modulename}};
    % beware: magic numbers here
    \draw (\x/2,2.05) node {\parbox[c][2\baselineskip][t]{15cm}{\disclaimer}};
    \draw (\x/2,3.2) node {\parbox[c][2\baselineskip][t]{15cm}{Your name (\textsc{block capitals})\dotfill}};
    \draw (\x/2,4.2) node {\parbox[c][2\baselineskip][t]{15cm}{Signature\dotfill}};
    \draw (\x/2,5.2) node {%
      \parbox[c][2\baselineskip][t]{9.2cm}{Personal tutor\dotfill\hspace*{1ex}}%
      \parbox[c][2\baselineskip][t]{5.8cm}{Date \dotfill}
    };
    \draw (\xright+1.5+0.4,\y/2+0.4) node{\Huge 10};
    \draw (\xright+1.5-0.7,\y/2+0.7) -- (\xright+1.5+0.7,\y/2-0.7);
    \draw (\xright+1.5,\y/2+1.1) node {\Large \textsc{total}};

    \draw [loosely dotted] (0,\y+0.15) -- (\x,\y+0.15);
    \draw [loosely dotted] (0,0-0.15) -- (\x,0-0.15);
    % \draw [loosely dotted] (0,0) -- (\x,0);
  \end{tikzpicture}\\[-0.05cm]
}%
\end{document}

%%% Local Variables: 
%%% mode: latex
%%% TeX-master: t
%%% End: 
